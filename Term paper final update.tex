\documentclass[12pt]{article}
\usepackage{graphicx}
\usepackage {hyperref}
\usepackage[left=3cm, right=3cm, top= 3cm]{geometry}
\title{\textbf{ECG database management system design}}
\author{Annpurna Matawale\\annpurnamatawale@gmail.com\\6th semester biomedical}
\date {  April 2022}



\begin{document}
\maketitle
\textbf{ABSTRACT}
\paragraph{}
\textbf{Databases} of electrocardiograms (ECGs) are used in medical research, pharmaceutical research, medical education, and health care. due to increasing needs in research, education, and health care Creating and maintaining ECG databases has been a challenge.It has evolved into a complicated issue.\paragraph{} This paper discusses the a fresh design strategy and a new application model
ECG Management System  (EMS). The  EMS isn't just for storing and managing data.Not only to collect ECG data, but also to automate the ECG workflow.to make the editing and checking of ECGs easier to acquire and compare the serial ECG recordings
save auditing data and generate a serial ECG report changes, and, above all, to communicate with Hospital information systems, for example, are examples of other systems.A \textbf{relational database manages }the patient demographic and clinical data, as well as ECGs. The system allows authorised users to access ECG data, ECG recordings, and ECG graphs.In data format, measurements and ECG interpretation eXtensible Markup Language is a markup language that allows you to express yourself in a variety of ways \textbf{(XML}),and to collect clinical cases for educational and research purposes queries written in a minimal structured \textbf{query language} In conclusion, the electronic  EMS is a powerful and versatile system.
a simple tool for research, education, and outreach
Finally, to improve patient care.

\paragraph{\textbf{Table of content}}
\paragraph{}1. Introduction \paragraph{} 2. Technical methods and materials
\paragraph{}3. ECG database management system
\paragraph{}4. Database schema
\paragraph{}5. File directory system
\paragraph{}6. Dataformate exchange
\paragraph{}7. web server interference

\paragraph{}8. System communication
\paragraph{}9. Application
\paragraph{}10. Conclusion
\paragraph{}11. References
\paragraph{\textbf{INTRODUCTION}}
\paragraph{} \textbf{ECG databases} and\textbf{ database management systems }are required for medical research, medical education, and patient care . Over the last few decades, many research institutions, healthcare organisations, and manufacturers have devised and constructed database management systems, but only a small percentage of them have been successful . The biggest challenge arises from the rapid advancement of technology.Current "state-of-the-art" technology is frequently obsolete within a few years, if not months. In an era of rapid clinical and information technology advancement, designing such a sophisticated ECG database management system has become increasingly difficult. The development and upkeep of a website consumes a significant amount of resources. Significant efforts have been made to the ECG management system as part of Philips Medical Systems' historical commitment to the medical community. The purpose of this study is to describe Philips ECG Management System's new design techniques, with a focus on its flexibility and openness in data access.
\paragraph{\textbf{TECHNICAL METHODS AND MATERIALS}}
\paragraph{\textbf{Materials}}
\paragraph{}Open source software was used to create the application. It is cost economical and flexible to design applications using open source software. MySQL is a database management system. It's a multi-user, multi-threaded SQL (Structured QueryLanguage) database server with a lot of power. The MySQL database server is built on a clever software architecture that prioritises speed and flexibility . MySQL's key advantages are its speed, robustness, and flexibility in storing huge things like ECGs. MySQL is also the database server of choice because it is gratis and its web interface is simple to create. As a web server, Apache is utilised. It's a feature-rich, fast, and stable open source web server .It's built to work on multitasking operating systems and can handle many requests at once. PHP is a scripting language that is widely used. PHP is an open source scripting language that is frequently used. Because PHP supports a wide range of databases and offers capabilities that link the Web and database environments, it is well-suited for web database applications.
\paragraph{\textbf{Methods}}
\paragraph{}Our web database's implementation can be represented as a three-tired architectural model. The database tier of the application is made up of MySQL, a database management system. The middle tier is built on top of the database tier and contains the majority of the application functionality, which is handled using PHP: Hypertext Preprocessor (PHP). The Apache web server is used to support tier-to-tier communication. On top of it, there's the client tier, which consists of standard web browser software that interacts with the application.
\paragraph{\textbf{ECG DATABASE MANAGEMENT SYSTEM (EMS)}}
\paragraph{} The system was implemented using a \textbf{three-tier architectural} approach, which has been an industry standard for decades and has been employed in numerous studies. With a web-based front-end interface, the database was built on \textbf{MySQL} and ran on Windows. The ECG data were maintained with a \textbf{MySQL database management system (DBMS) }.\\ The built-in file-directory system in Windows was used to store temporal data and ECG waveform data. For system implementation, open source software was chosen because of its low maintenance costs, reliability, and efficiency. \textbf{Apache, PHP, and MySQL }served as the system's backbone, all of which were freeware but performed admirably. The MySQL DBMS and a file-directory system formed the system's foundation (first-tier). To accommodate the MySQL database and PHP scripts, the file-directory structure was changed. The second tier, which was built with Apache and PHP, was built on top of the first. This was the primary component of the system. To make communication between the subsystems easier, Apache was employed as the web server. PHP was the preprocessor for web application logic interpretation.\\ Commands for each sub-system were interpreted and processed by PHP and then provided to the Apache server for web representation. Finally, web browsers served as the Internet's third tier for end-users (clients).


\noindent

three tier architecture of EMS is shown in Figure~\ref{three tier architecture}. \\
\begin{figure}[h]
\centering
\includegraphics [scale=0.2]{three tier architecture.jpg}
\caption{the three tier architecture of web based electronic ECG management system}
 \label{three tier architecture}
\end{figure}
.
\paragraph{\textbf{1. First tier}}
\paragraph{}  The MySQL DBMS and a file-directory system formed the system’s foundation
(first-tier). To accommodate the MySQL database and PHP scripts, the file-directory
structure was changed. 
\paragraph{\textbf{2. Second tier}}
\paragraph{}The majority of the application logic is handled by the second tier. It primarily serves as a link between the other two layers. An Apache web server, the PHP Zend engine (scripting engine), and PHP scripts make up the system. The data from the user is collected by the second tier (a request is made to the web server, which is then passed to the Zend engine (scripting engine) via the Zend engine's web server interface). A data query is constructed and submitted to the database based on the information provided by the user (An appropriate PHP script is retrieved from the disc which is then compiled and run by the Zend engine).Finally, the results are prepared and provided to the user through the Internet for viewing. (The output of the web server interface is returned to the web server, which then sends it on to the user.)
\paragraph{\textbf{3. Third tier}}
\paragraph{} Finally, web browsers served as the
Internet’s third tier for end-users (clients).web browsers are very thin clients, using web browser as thin clients offer the advantage of easy deployment, and makes our application platform independant .
\paragraph{\textbf{DATABASE SCHEMA }}
\paragraph{}ECG database system is a \textbf{relational database management system} RDBMS.EMS database composed of normalized tables containing digital ECG waveform , patients demographics , measurement and ECG interpretation. The majority of ECG data is available through individual columns in relational tables. For information retrieval, this approach allows for the use of simple to complicated SQL queries.\\Clinics, phones, devices, appointments, ecg test, ecg raw data, nurses, patient information, and verification code are the nine tables that make up our database.
The clinic table keeps track of information such as the clinic's address, phone number, name, and email address.
\noindent

schema of  EMS is shown in Figure~\ref{database schema}. \\
\begin{figure}[h]
\centering
\includegraphics [scale=0.5]{database schema.jpg}
\caption{schema of ECG database system}
 \label{database schema}
\end{figure}
.

 The responsibilities of RDBMS is  : \\ • Make effective data storage\\ • Provide data integrity constraints\\ • Provide concurrent multi-user access\\ • Ensure permitted and secure data access\\ • Recover the database in the event of a crash
 \paragraph{\textbf{FILE DIRECTORY SYSTEM}}
 Only the index/indices to individual files were stored in the database for the database system's flexibility and performance. Using an ASCII-based file format (.tx0,. tx1, .tx2). large portions of biomedical data were stored in different files. Waveform data generated from 2.5 seconds standard 12-lead rhythm and 10 seconds long-lead rhythm were stored in filename.tx1 and filename.tx2, respectively, on the Windows operating system. The database system kept track of each file's path. The hospital and department codes were kept in the database for future inter-hospital data interchange. The decoded files (tal. 1x2) and related files (such as .png, .xml, .svg, and so on) were saved in sub-directories. The file-directory system was organised in a tree, as shown in Figure.
 \noindent

 structure of file directory system is shown in Figure~\ref{file directory}. \\
\begin{figure}[h]
\centering
\includegraphics [scale=0.5]{file directory.jpg}
\caption{structure of file directory system}
 \label{file directory}
\end{figure}
.
\paragraph{\textbf{DATAFORMATE EXCHANGE}}

\paragraph{} The raw ECG data was saved in ASCII format in EMS. Several open-source ECG formats, such as XML (FDA-XML or ecgML), SVG, and PNG, were utilised to communicate ECG data in order to suit the needs of clinical physicians and academic researchers.\\ Four open sources ECG related formats were generated by this system including FDA XML, ecgML, SVG and PNG.
\paragraph{\textbf{Rendering ECG with XML}}
\paragraph{}Since November 2001, the Food and Drug Administration (FDA) has advocated an XML-based format for annotated ECG waveform data. Using the HL7 Version 3 standard, an FDA XML format was created.ecgML is an alternative XML format for ECG that was created by Wang et al and has a simpler structure.The system stores ECGs in an industry-standard XML format. The most major move done by Philips Medical System is the availability of the XML data format schema to consumers. This new strategy allows users to have the most access to the data stored while also promoting system compatibility.
\paragraph{\textbf{Rendering ECG with SVG}}
\paragraph{}Some graphic formats for viewing and transmitting ECG waveform data were utilised to represent it on the web. The most prevalent formats are SVG and PNG, which have been endorsed by the World Wide Web Consortium (W3C). PNG was chosen as the master image output format and SVG was chosen as the scalable vector image output format for this investigation. The W3C website and the internet have extensive documentation and support for these two formats.\\The system also generated a PNG file online. PNG is supported by almost all web browsers, including Netscape and Internet Explorer. Authorized end-users can review the ECG records saved in the database using the PNG format from anywhere in the world.
\paragraph{}The results of the signal analysis can be included in the XML-related ECG files because FDA XML and ecgML both permit annotation of an ECG waveform. The study ECG management system's functionality is improved by integrating these analytic modules with other components.
\paragraph{\textbf{WEB SERVER INTERFERENCE}}
\paragraph{}WebECG, a user-friendly and simple-to-use web interface, was created. End-users who have been granted access to the system can log in and search the database. The EMS system's query function is shown in the diagram.
\noindent

Query function of the EMS system is shown in Figure~\ref{three tier architecture}. \\
\begin{figure}[h]
\centering
\includegraphics [scale=1]{ECG query.jpg}
\caption{Query function of EMS system}
 \label{ECG query}
\end{figure}
.
An end-user can query the database in one of two ways: (1) by patient id, or (2) by date. 
\paragraph{\textbf{Specific disease database}}
\paragraph{}Three particular disease databases have been built using clinical data during the last two years.

(1) Hyperkalemma: There are 67 records in this category.\\ (2) Hypokalemia (80 cases):

(3) AMI-98 documents\\ Clinical physicians confirmed the diagnosis after each patient was diagnosed. The findings of laboratory tests were used to collect related clinical data and laboratory investigations, such as electrolyte concentrations (sodium and potassium for hypernephrosis and hypokalemia), and cardiac enzymes (CPK, MB, and Troponin-1 for AMI).
\paragraph{\textbf{SYSTEM COMMUNICATION}}
\paragraph{}TCP/IP over LAN and WAN, as well as transmission control protocol/point-to-point protocol (TCP/PPP) over modem and dialup connections, are supported for communication with electrocardiographs for delivery of ECGs to the Philips EMS for storage and subsequent evaluation.
Unicode encoded XML and HL7 message exchange are used to allow bidirectional interactions between the EMS and the hospital information system (HIS).
ECG orders are typically imported from the HIS Data Order Entry system, ECG order fulfilment is exported to the HIS billing system, and validated ECG readings are exported to the enterprise clinical information system.
\noindent

system communication is shown in Figure~\ref{communication system}. \\
\begin{figure}[h]
\centering
\includegraphics [scale=0.5]{system communication.jpg}
\caption{communication between the ECG management system and the ECG devices is shown in this diagram. bidirectional communication requirement increase the complexity significantly }
 \label{system communication}
\end{figure}


\pagebreak
\paragraph{\textbf{APPLICATION}}

\paragraph{}Medical research, medical education, and patient care are all emphasised in the application models. The most recent design incorporates a number of new application models.
\paragraph{\textbf{1. Medical pharmaceutical research}}
\paragraph{}Over the last few decades, there has been debate about who manages the ECG database on a commercial management system. The openness of the industry standard XML format, which was a fundamental breakthrough design strategy in the Philips EMS, allows researchers to keep control of the ECG information while facilitating access. Propritary ECG format issues are no longer an issue. Researchers, cardiologists, and other healthcare experts will benefit greatly from the system's architecture. In medical and pharmaceutical studies, researchers have complete flexibility in data storage and retrieval, as well as in accessing and searching ECG by measurements, interpretation, or the high-fidelity digital waveform.
\paragraph{\textbf{2.Medical Education}}
\paragraph{}In medical science, reading an ECG is an art. ECG examples in medical school texts, on the other hand, are scarce. Training ECG reading might be challenging without an ECG database management system. ECG examples may be searched, retrieved, shown, and printed in seconds using the Philips EMS. Using a SQL editor and tools, you may easily query the database. Almost all ECG data is provided in relational tables as discrete columns. Virtually any search query can be submitted to the system using SQL's powerful commands such as "SELECT" and "JOIN," and results can be returned rapidly.
\paragraph{\textbf{3. ECG management in patient care}}
\paragraph{}In recent years, most hospitals in the United States have faced a scarcity of skilled ECG professionals. To compensate for the limitation of resources, the Philips EMS automates the administration of ECG orders and the reporting of order results. From downloading ECG orders from the HIS order entry system to uploading ECG order fulfilment data to the HIS billing system to sending ECG confirmed results to the enterprise clinical information system, the flexible workflow model and order handling capability work together to maximise process automation. All of the foregoing makes it easier to work with ECG data and makes it easier to make patient-care decisions. Serial ECG comparisons are another advanced application available on the ECG management system. Cardiologists can use the serial comparison technique to diagnose ECGs and track the progression and regression of a cardiac condition.
\paragraph{\textbf{CONCLUSION}}
\paragraph{} A web-based electronic ECG administration system was developed that provides multifarious ECG management solutions such as stability, compatibility, accessibility, adaptability, user-friendly interface, and easy-to-maintain features. To save maintenance costs, open source programmes with good performance and reliability, such as Apache, PHP, and MySQL, were employed.\\Finally, the created electronic ECG management system can provide useful ECG informatics services, such as online ECG pattern detection for clinical practitioners and ECG signal processing and analysis for researchers. This web-based electronic system's design and development are critical components in offering better and more efficient online ECG data handling. With the inclusion of signal analysing modules, the system can be used as a diagnostic tool for physicians as well as a research platform.\\The format-exchanging programmes created in this study have the potential to improve ECG data interoperability. Other medical signals include holter ECGs, exercise ECGs, and patient monitors, among others. In the future, the phenocardiogram could be investigated: Intra-hospital integration systems, such as communication with the HIS, interaction with the Picture Archiving and Communication System (PACS), and integration with the Electronic Patient Record, are all possible research topics (EPR) The goal should eventually be to create a country-wide inter-hospital medical signal database.
\paragraph{\textbf{REFERENCE}}
\paragraph{} 

[1] Norman JE, Bailey JJ, Berson AS, et al: NHLBI workshop on the utilization of ECG databases: preservation and use of existing ECG databases and development of future resources. J Electrocardiol,
1998;31(2):83-89.\\

[2] Chiarugi F, Lombardi D, Lees PJ, et al: Support of daily
ECG procedures in a cardiology department via the integration of an existing clinical database and a commercial ECG management system. Ann Noninvasive Electrocardiol, 2002:7(3):263-270.\\

[3] Nose Y, Akazawa K, Yokota M, et al.: An electrocardiogram database incorporated into the hospital information system. Med Inform (Lond), 1987;12(1):1-9.







\end{document}