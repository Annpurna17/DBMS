\documentclass[12pt]{article}
\usepackage[left=3cm, right=3cm, top= 3cm]{geometry}
\title{\textbf{ECG database management system design}}
\author{Annpurna Matawale\\annpurnamatawale\\6th semester biomedical}
\date {  january 2022}



\begin{document}
\maketitle
\textbf{ABSTRACT}
\paragraph{}
\textbf{Databases} of electrocardiograms (ECGs) are used in medical research, pharmaceutical research, medical education, and health care. due to increasing needs in research, education, and health care Creating and maintaining ECG databases has been a challenge.It has evolved into a complicated issue.\paragraph{} This paper discusses the a fresh design strategy and a new application model
ECG Management System by Philips (EMS). The Philips EMS isn't just for storing and managing data.Not only to collect ECG data, but also to automate the ECG workflow.to make the editing and checking of ECGs easier to acquire and compare the serial ECG recordings
save auditing data and generate a serial ECG report changes, and, above all, to communicate with Hospital information systems, for example, are examples of other systems.A \textbf{relational database manages }the patient demographic and clinical data, as well as ECGs. The system allows authorised users to access ECG data, ECG recordings, and ECG graphs.In data format, measurements and ECG interpretation eXtensible Markup Language is a markup language that allows you to express yourself in a variety of ways \textbf{(XML}),and to collect clinical cases for educational and research purposes queries written in a minimal structured \textbf{query language} In conclusion, the Philips EMS is a powerful and versatile system.
a simple tool for research, education, and outreach
Finally, to improve patient care.

\paragraph{\textbf{Table of content}}
\paragraph{}1. Introduction \paragraph{} 2. Technical methods and materials
\paragraph{}3. ECG database management system
\paragraph{}4. Database schema
\paragraph{}5. Dataformate exchange
\paragraph{}6. Workflow
\paragraph{}7. System communication
\paragraph{}8. Application
\paragraph{}9. Conclusion
\paragraph{}10. References
\paragraph{\textbf{INTRODUCTION}}
\paragraph{} \textbf{ECG databases} and\textbf{ database management systems }are required for medical research, medical education, and patient care . Over the last few decades, many research institutions, healthcare organisations, and manufacturers have devised and constructed database management systems, but only a small percentage of them have been successful . The biggest challenge arises from the rapid advancement of technology.Current "state-of-the-art" technology is frequently obsolete within a few years, if not months. In an era of rapid clinical and information technology advancement, designing such a sophisticated ECG database management system has become increasingly difficult. The development and upkeep of a website consumes a significant amount of resources. Significant efforts have been made to the ECG management system as part of Philips Medical Systems' historical commitment to the medical community. The purpose of this study is to describe Philips ECG Management System's new design techniques, with a focus on its flexibility and openness in data access.






\end{document}