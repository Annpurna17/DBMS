\documentclass[12pt]{article}
\usepackage[left=3cm, right=3cm, top= 3cm]{geometry}
\title{\textbf{ECG database management system design}}
\author{Annpurna Matawale\\annpurnamatawale\\6th semester biomedical}
\date {  january 2022}



\begin{document}
\maketitle
\textbf{ABSTRACT}
\paragraph{}
Databases of electrocardiograms (ECGs) are used in medical research, pharmaceutical research, medical education, and health care. due to increasing needs in research, education, and health care Creating and maintaining ECG databases has been a challenge.It has evolved into a complicated issue.\paragraph{} This paper discusses the a fresh design strategy and a new application model
ECG Management System by Philips (EMS). The Philips EMS isn't just for storing and managing data.Not only to collect ECG data, but also to automate the ECG workflow.to make the editing and checking of ECGs easier to acquire and compare the serial ECG recordings
save auditing data and generate a serial ECG report changes, and, above all, to communicate with Hospital information systems, for example, are examples of other systems.A relational database manages the patient demographic and clinical data, as well as ECGs. The system allows authorised users to access ECG data, ECG recordings, and ECG graphs.In data format, measurements and ECG interpretation eXtensible Markup Language is a markup language that allows you to express yourself in a variety of ways (XML),and to collect clinical cases for educational and research purposes queries written in a minimal structured query language In conclusion, the Philips EMS is a powerful and versatile system.
a simple tool for research, education, and outreach
Finally, to improve patient care.








\end{document}